\subsection{BRKGA implementation}

\subsubsection{Chromosome structure}

Since in this problem all the nurses are equal in terms of schedule, to exploit the diversification of the chromosoem we have chosen between two approaches. The firsts is to assign each gene of the chromosome in order to each hour of the schedule (chromosome length $hours$). This ways we can sort the hours of the schedule and assign nurses to hours in that order. The other choice is instead to define an excess of working nurses for each hours, that is, to increase the demand of each hour randomly by the chromosome. We choose the second approach as it appears to diversify more in the initial tests that we have performed (around 20\% of different fitness between individuals in each breed versus only around 5\% for the first approach).


\begin{algorithm}[H]
\KwIn{chromosome, hours, demand, nNurses}
%\Parameter{Some parameter}
\KwOut{demand}    

\SetKwData{Left}{left}
\SetKwData{This}{this}
\SetKwData{Up}{up}
\SetKwFunction{Union}{Union}
\SetKwFunction{FindCompress}{FindCompress}

i = 0 \\
\ForEach{$gene$ $\in$ $chromosome$}{

	\If{$gene < 0.2$}{
		$new\_hourly\_demand = ceil(demand_{i} \cdot 0.8 \cdot nNurses)  $ \\
		$demand_{i} = new\_hourly\_demand$ \\
	}
	$ i += 1$\\

}

$\textbf{return}$ demand
\caption{BRKGA Decoding algorithm}\label{brkga.decoding}
\end{algorithm}


\subsubsection{Decoder}


\begin{algorithm}[H]
\KwIn{population, nNurses, hours, demand, constraints}
%\Parameter{Some parameter}
\KwOut{population}    

\SetKwData{Left}{left}
\SetKwData{This}{this}
\SetKwData{Up}{up}
\SetKwFunction{Union}{Union}
\SetKwFunction{FindCompress}{FindCompress}

\ForEach{$individual$ $\in$ $population$}{
new\_demand $\leftarrow decoding(individual, hours, demand, nNurses, constraints)$ \\
population $\leftarrow assignNurses(solution, nNurses, hours, new\_demand)$ \\
}
$\textbf{return}$ population
\caption{BRKGA Decoder algorithm}\label{BRKGA.decoder.mainLoop}
\end{algorithm}

As illustrated in $Algorithm$ \ref{BRKGA.decoder.mainLoop}, the decoder simply decodes each individual chromosome with $Algorithm$ \ref{brkga.decoding}, and calls $assignNurses$ to assign the nurses according to the newly computed $new\_demand$. Each solution is stored in the population with its corresponding fitness already computed in the $assignNurses$ function ($Algorithm$ \ref{BRKGA.assignNurses}).

\begin{algorithm}[H]
\KwIn{solution, nNurses, hours, new\_demand, constraints}
%\Parameter{Some parameter}
\KwOut{solution}    

\SetKwData{Left}{left}
\SetKwData{This}{this}
\SetKwData{Up}{up}
\SetKwFunction{Union}{Union}
\SetKwFunction{FindCompress}{FindCompress}


\ForEach{$h \in hours$}{
	$mustWorkList, canWorkList = computeCandidateAssignments(solution, h, constraints)$\\
	\ForEach{$n \in mustWorkList$}{
		$ assignWorkingHour(solution, n, h)$\\
		$ demand_{h} \leftarrow updateRemainingDemand(solution,h)$\\
	}
	\ForEach{$n \in canWorkList$}{
		\If{$demand_{h} > 0$}{
			$ assignWorkingHour(solution, n, h)$\\
			$ demand_{h} \leftarrow updateRemainingDemand(solution,h)$\\
		}
	}
}

\If{$Feasible(solution)$}{
	$solution(fitness) \leftarrow computeFitness(solution)$ \\
}
\Else{
	$solution(fitness) \leftarrow \infty $
}

$\textbf{return}$ solution
\caption{assignNurses}\label{BRKGA.assignNurses}
\end{algorithm}


As shown in $Algorithm$ \ref{BRKGA.assignNurses}, this algorithm assigns nurses to the schedule according to the demand and the constraints. For each possible hour in the schedule, in order, the algorithm selects the nurses that should work at this specific hour according to the constraints, and the nurses that could work at this specific hour according to the constraints (line 2). The function $computeCandidateAssignments$ walks through all the nurses, assigning the current hour and verifying if the nurse schedule is valid according to the constraints. From that "canWorkList", it walks through all the nurses from that list, removing the assignment in the specific hour and verifying if the schedule would be valid or not. That way it creates the second "mustWorkList" with nurses that must work to have valid schedules, removing them from the first list in order to have two disjoint sets. Once the two lists are set, it first assigns all the nurses that must work to have a valid schedule (lines 3 to 5). Each time, the remaining demand is updated (line 5). Then it walks throught the list of nurses that can work and assign the hour if and only if the remaining demand in that specific hour is positive (line 8). The final solution is then updated with its fitness, which is the same as the cost or number of nurses that work any hour during the schedule. If the solution is not feasible, that means that the demand is not fulfilled in each hour, then the fitness is set to be infinite or big enough.

