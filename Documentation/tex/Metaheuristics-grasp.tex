\subsection{GRASP implementation}

\subsubsection{Constructive phase}

The GRASP meta-heuristic has a constructive phase that is concerned to build up a feasible solution. This phase can be deterministic or include a certain amount of randomness by controlling a paremeter value $\alpha$.  This means
that for every execution, different solution could emerge. We use the basic GRASP construction phase described in \cite{grasp}. The specific parts of the implementation are the initialization of the candidate set $C$, implemented in the function $initializeCandidates$ depicted in $Algorithm$ $1$, and the greedy cost function shown in the next subsection.

\begin{algorithm}[H]
\KwIn{nNurses, hours, schedule\_constraints}
%\Parameter{Some parameter}
\KwOut{candidate list C}    

\SetKwData{Left}{left}
\SetKwData{This}{this}
\SetKwData{Up}{up}
\SetKwFunction{Union}{Union}
\SetKwFunction{FindCompress}{FindCompress}

$C$ $\leftarrow$ $\emptyset$ \\
\ForEach{$h \in hours$}{
	
	$E$ $\leftarrow$ $initializeEmptySchedules(10, hours, schedule_constraints)$ \\
	\ForEach{$hindex \in [h+1, hours]$}{ 
		$modulo\_param = 2$ \\
		\ForEach{$e \in E$}{
			\If{$hindex - h \bmod  modulo\_param > 0$}{
				addWorkingHour(e, hindex) \\
				\If{notValidConstraints(e)}{
					removeWorkingHour(e,hindex) \\
					addRestingHour(e, hindex)  \\
				}
			}
			\Else{
				addRestingHour(e, hindex) \\
				\If{notValidConstraints(e)}{
					removeRestigHour(e,hindex) \\
					addWorkingHour(e, hindex) \\
				}
			}

			\If{notValidConstraints(e, hindex - h)}{

				$E$ $\leftarrow$ $E \cap e$ \\
			}
			$modulo\_param += 1$
		}
	}
	$C$ $\leftarrow$ $C \cup E$
} 
$\textbf{return}$ <$C$>
\caption{initializeCandidates}\label{alg.mainLoop}
\end{algorithm}


The ComputeCandidateElements function takes as input the total number of nurses, the number of hours to schedule and the rest of the constraints(maximum presence hours, consecutive hours, total hours and minimum hours). The output is a list of multiple schedules that each nurse can be assigned to. A schedule is the list of hours in which a nurse works must work.\\
First, the algorithm initializes an empty candidate set. 
Then, for each hour in the schedule, it creates 10 different types of schedules beginning a this specific hour(line3). The difference between the 10 types of schedules is the compactness of the working hours. This is controlled by a parameter used to do the modulo with current hour index in the built schedule. This allows to create from the most compact schedule with all working hours consecutive until the constraints allows to do it ($hindex - h \bmod hours + 1$), to the most sparse schedule consisting of alternating working and resting hours (using $ hindex -h  \bmod 2$).
The different schedules, started at different hours are built incrementally by adding work or rest hours depending on the modulo parameter (line 7), and always taking into consideration the validity of the resulting partial schedule (line 9, 16), in which case the validity is temptatively fixed. If no more options remain and the schedule becomes invalid for the constraints of the problem(line 21), it is removed from the set $E$ (line 22) and so it is not later saved to the candidate set $C$ (line 27).

\subsubsection{Greedy cost function}

As all the nurses are equal in this scenario, there's only one thing that the greedy cost function can determine, the number of hours of demand that a single nurse schedule contributes to. We are able to do this because all schedule candidates produced are valid (they follow the constraints). We conside $e$ to be a candidate schedule for a single nurse, being $e_{h} = 1$ if the nurse has to work or 0 otherwise. We also use a big constant number K, that should be bigger than the value $hours$ (for example $nNurses$) and we consider $remaining\_demand_h$ to be the demand that is not covered by any other schedule that is present in the solution at the hour $h$	.

\begin{center}
 $ gc(e) = K - \sum_{h=1}^{Hours} remaining\_demand_h \times e_{h}$
\end{center}

\subsubsection{Other problem specific details}

There are some specific modifications of the GRASP constructie phase of \cite{grasp} applied to this problem. After updating the candidate set, in \cite{grasp} page 2, we test the feasibility of the updated solution in each iteration of the constructive phase, and in the case we are having a feasible solution, we leave the loop. If not, we continue looping until no candidate schedules are available or no more candidate schedules can be added (all nurses have a schedule). A solution is feasible if for each hour, the demand is less or equal to the number of nurses working at this hour. Another improvement introduced in the basic version of the algorithm, is the fact that instead of removing candidate elements(single nurse schedules) from the candidate set and adding them to the solution, we generate a basic list of possible candidates and each time a nurse is assigned one of them, we don't remove it from the candidate set. What we do is recompute each time the greedy cost of each candidate when the solution is updated. This is possible thanks to the fact that all nurses are equal under the problem assumptions and constraints. That way we reduce the number of candidates that we have to generate and sort.



\subsubsection{Local Search}

Once the constructive phase ends, the local search tries to improve the given solution by looking at its neighbourhood,i.e. the near solutions will be searched. We show the pseudocode in ($Algorithm$ $2$). We perform a lightweight local search after every GRASP constructive phase, and then when $maxIterations$ of GRASP constructive phases have been performed, we execute a more intensive local search.

\begin{algorithm}[H]
\KwIn{solution, nNurses, hours, demand}
%\Parameter{Some parameter}
\KwOut{solution}    

\SetKwData{Left}{left}
\SetKwData{This}{this}
\SetKwData{Up}{up}
\SetKwFunction{Union}{Union}
\SetKwFunction{FindCompress}{FindCompress}

$sol \leftarrow removeExceedingWorkingHours(solution)$ \\
improved = True \\
\While{improved}{
	
	improved = False \\
	$G \leftarrow findValidScheduleGaps(sol)$ \\
	\While{ $G \neq \emptyset $}{
		\ForEach{$n \in nNurses $}{
			$sol' \leftarrow sol$ \\
			$G_{aux} \leftarrow G$ \\
			\ForEach{$h \in hours$}{
				\If{$sol'(n,h) \neq 1$ }{
					continue \\
				}
				$sol' \leftarrow swapHourAssignment(sol', n, G(h), h) $ \\
				$G \leftarrow removeFirstScheduleGapAtHourH(G,h) $ \\
				
			}

			\If{$isValidSolution(sol')$ and $improvesCost(sol', sol)$}{
				$sol \leftarrow sol' $ \\
				Improved = True
			}\Else{
				$G \leftarrow G_{aux}$ \\
			}
		}
	}
}


solution $\leftarrow$ sol\\
$\textbf{return}$ solution
\caption{Local Search}\label{alg.mainLoop}
\end{algorithm}

The local search algorithm starts by removing any work assignment that is not needed to fulfill the demand of the problem (line 1). This will make room for later swaping of hour assignments (and sometimes frees some nurses). Then we initialize the improvement while loop, a loop that will stop when no nurse can be freed anymore. Inside the loop(line 5), we first save all the gaps of the solution. That means all hours that a working nurse is resting but that he/she could work and its schedule would still be valid. The gap list $G$ contains for each possible hour of the schedule, a list of nurses that have a valid gap at these specific hour. A new loop is created until there's no more valid gap left (line 6). Inside this loop, we try, for each nurse, to assign its working hours to the first nurse that is listed in the gap list for each hour(line 14, 15).  
If after swaping all the hours of a nurse the resulting solution is valid and improves the solution, then the solution is updated (line 17). If not, we undo all the changes to the gap list (line 22) and we don't update the current solution. At the end, the returned solution contains either the same solution as the beginning, or a new solution with less working nurses but all valid and fullfiling the demand at each hour.

To transform the firts improvement local search shown previously into an intensive local search,  we can add room for performing iterations of the local search even if the solution has not improved. We could limit that with a variable called $failedIterations$. Thats the approach we choose to perform on the last localsearch we perform after the main GRASP loop (see first algorithm in \cite{grasp} on page 2).
Usually, there's also the possibility to implement a first improvement local search and a best improvement local search. There is no such possibility with this implementation. We would require to start swaping nurse hour assignments from different nurses instead of following always the same order. This turned to be too costly in terms of algorithmic cost and we decided to go with this simpler approach.



