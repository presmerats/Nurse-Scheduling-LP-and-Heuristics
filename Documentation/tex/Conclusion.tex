
In this project we developed an ILP model and two meta-heuriscs in order to find out the optimal (for ILP) and good (for meta-heuristics) nurse schedules in a hospital taking into account several constraints.

On the first hand, the Integer Linear Programming part has been really challenging to end up with a really simplified model that takes into account every constraint. We also saw how the cost of solving a problem increases exponentially with the problem size, whereas with the meta-heuristic models the cost increases much more slowly allowing use to solve much more bigger instances.

On the second hand, the meta-heuristics has been also a challeng to build up with a reasonable greedy function, that is in fact the core of all applied meta-heuristics here. But the more challenging part has been to get a good set of inputs for the experiments, specially the largest of them. It has been so difficult since the complexity is not only a matter of number of nurses but also another factors that determine the problem context, but finally we end up with good inputs by tunning the instance generator. It is also difficult to set the parameters of the meta-heuristic algorithms to exploit the diversification and intensification options that they poses.

Finally, we demonstrated how simpler algorithms can be combined together to attack difficult problems and obtain a satisfactorial solution.

\pagebreak